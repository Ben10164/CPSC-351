\documentclass{article}
\usepackage{amssymb}

\title{Project 1}
\author{Jesse Adams, Dominic Orsi, Ben Puryear}
\date{September 7th, 2023}

\begin{document}

\maketitle

\section{Write a short, informal English description for this set: \{1,3,5,7, …,\}.}
\begin{center}
	The set \{1,3,5,7,...,\} contains all positive odd integers.
\end{center}

\section{Write a formal description of this set: The set containing all integers that are greater than 5.}
\begin{center}
	$\{x \in \mathbb{Z} | x \ge 5 \}$
\end{center}

\section{Let $A = \{x,y,z\}$ and $B = \{x,y\}$}
\subsection{What is $A$ x $B$, where x is the Cartesian product operator?}
\begin{center}
	\begin{tabular}{ |c|c|c| }
		  & x     & y     \\
		\hline
		x & (x,x) & (x,y) \\
		y & (y,x) & (y,y) \\
		z & (z,x) & (z,y) \\
	\end{tabular}
\end{center}

\section{Given sets $A$ and $B$ from problem 3, what is the power set of $B$?}
$B = \{x,y\}$
\begin{center}
	The power set of $B$ is $\{0, \{x\},\{y\}, \{x,y\}\}$
\end{center}


\section{Let $X = \{1,2,3,4,5\}$ and $Y = \{6,7,8,9,10\}$}
Here at two functions, $g$ and $f$, defined in the following tables, over $X$ and $Y$:
\begin{center}
	\begin{tabular}{ |c|c|c|c|c|c| }
		\hline
		g & 6  & 7  & 8  & 9  & 10 \\
		\hline
		1 & 10 & 10 & 10 & 10 & 10 \\
		2 & 7  & 8  & 9  & 10 & 6  \\
		3 & 7  & 7  & 8  & 8  & 9  \\
		4 & 9  & 8  & 7  & 6  & 10 \\
		5 & 6  & 6  & 6  & 6  & 6  \\
		\hline
	\end{tabular}

	\begin{tabular}{ |c|c| }
		\hline
		n & f(n) \\
		\hline
		1 & 6    \\
		2 & 7    \\
		3 & 6    \\
		4 & 7    \\
		5 & 6    \\
		\hline
	\end{tabular}
\end{center}
\subsection{What is the value of f(2)?}
\begin{center}
	7
\end{center}
\subsection{What is the range and domain of f?}
\begin{center}
	Domain: [1, 5] \\
	Range: [6, 7]
\end{center}
\subsection{What is the value of g(2,10)?}
\begin{center}
	6
\end{center}
\subsection{What is the range and domain of g?}
\begin{center}
	Domain: [1, 10] \\
	Range: [6, 10]
\end{center}
\subsection{What is the value of g(4,f(4))?}
\begin{center}
	8
\end{center}

\section{Prove $2 = 1$}
Basis: Consider the equation $a = b$
\begin{enumerate}
	\item Multiply both sides by a to obtain $a^{2} = ab$
	\item Subtract b2 from both sides to obtain $a^2 - b2 = ab - b^2$
	\item Factor each side, to obtain $(a-b)(a+b) = b(a-b)$
	\item Divide by $(a - b)$ to get $(a+b) = b$
	\item Now, let $a = b = 1$.
	\item Therefore $2 = 1$, which is what we set out to prove.
\end{enumerate}
There seems to be an error here.  What is it?

\begin{center}
	The error is occurs in step 4.
	This step involves dividing both sides of the equation by (a-b). However, this is impossible because the basis for this proof states that a=b, making dividing by (a-b) be dividing by zero. Because of its undefined properties, it is not mathematically possible to divide by zero.
\end{center}

\section{Let $(n) = 1 + 2 + ... + n$ be the sum of the first n positive integers. Using PMI, prove that: $S(n) = \frac{1}{2}n(n+1)$}

Prove:
$$S(n) = \frac{1}{2}n(n+1)$$
$$\frac{1}{2}n(n+1) = \frac{n(n+1)}{2}$$
Base Case:
$$S(1) = \frac{1(1+1)}{2}$$
$$\frac{1(1+1)}{2} = 1$$
$$S(1) = 1$$
Induction Step:

Assume true for $k$:
$$S(k) = \frac{k(k+1)}{2}$$

Define $S(k+1)$
$$S(k+1) = 1 + 2 + 3 + ... + k + (k+1)$$

Add $(k+1)$ to the defined version of $S(k)$
$$\frac{k(k+1)}{2}+(k+1)$$
% $$S(k+1) = \frac{k(k+1)}{2}+(k+1)$$

Multiply $(k+1)$ by $\frac{2}{2}$ to equalize the base
$$\frac{k(k+1)}{2}+\frac{2(k+1)}{2}$$
% $$\frac{k(k+1)}{2}+(k+1) = \frac{k(k+1)}{2}+\frac{2(k+1)}{2}$$

Evaluate the addition
$$\frac{k(k+1)+2(k+1)}{2}$$
% $$\frac{k(k+1)}{2}+\frac{2(k+1)}{2} = \frac{k(k+1)+2(k+1)}{2}$$

Factor the numerator
$$\frac{(k+1)(k+2)}{2}$$
% $$\frac{k(k+1)+2(k+1)}{2} = \frac{(k+1)(k+2)}{2}$$

Expand the addition in $(k+2)$
$$\frac{(k+1)((k+1)+1)}{2}$$
% $$\frac{(k+1)(k+2)}{2} = \frac{(k+1)((k+1)+1)}{2}$$

Now the inducted equation matches the defined $S(k+1)$
$$\frac{(k+1)((k+1)+1)}{2} = S(k+1)$$

\section{Let $C(n) = 13 + 23 + ... + n^3$ be the sum of the cubes of the first n positive integers. Using PMI, prove that: $C(n)$ has this closed form equivalent:$\frac{1}{4}(n^4+2n^3+n^2)$}

Prove:
$$C(n)=\frac{1}{4}(n^4+2n^3+n^2)$$
$$\frac{1}{4}(n^4+2n^3+n^2)=\frac{n^4+2n^3+n^2}{4}$$
Base Case:
$$C(1)= \frac{1^4+2(1)^3+1^2}{4}$$
$$\frac{1^4+2(1)^3+1^2}{4} = 1$$
$$C(1) = 1$$
Induction Step:

Assume true for some $k$
$$C(k)=\frac{k^4+2k^3+k^2}{4}$$

Add $(k+1)^3$ for the next step in the equation
$$\frac{k^4+2k^3+k^2}{4} + (k+1)^3$$

Expand $(k+1)^3$
$$\frac{k^4+2k^3+k^2}{4} + ((k+1)(k+1)^2)$$
$$\frac{k^4+2k^3+k^2}{4} + ((k+1)(k^2+2k+1))$$
$$\frac{k^4+2k^3+k^2}{4} + ((k^3+2k^2+k)+(k^2+2k+1))$$
$$\frac{k^4+2k^3+k^2}{4} + (k^3+3k^2+k+1)$$

Multiply the RHS by $\frac{4}{4}$ to equalize the base
$$\frac{k^4+2k^3+k^2}{4} + \frac{4(k^3+3k^2+k+1)}{4}$$
$$\frac{k^4+2k^3+k^2}{4} + \frac{4k^3+12k^2+4k+4}{4}$$

Evaluate the addition
$$\frac{(k^4+2k^3+k^2)+(4k^3+12k^2+4k+4)}{4}$$
$$\frac{k^4+6k^3+13k^2+4k+4}{4}$$

Set the inducted equation equal to the assumed true equation for $C(k+1)$
$$\frac{k^4+6k^3+13k^2+4k+4}{4} = \frac{(k+1)^4+2(k+1)^3+(k+1)^2}{4}$$

Expand the RHS of the equation
$$[...]$$
$$\frac{k^4+6k^3+13k^2+4k+4}{4} = \frac{k^4+6k^3+13k^2+4k+4}{4}$$
\\
We have now proved that the inducted equation works for C(1), C(k), and C(k+1), therefore C(n) has been proven a true equation through mathematical induction

\end{document}