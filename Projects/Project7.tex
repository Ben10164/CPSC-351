\documentclass{article}
\usepackage{amsmath}

\title{Project 7}
\author{Jesse Adams, Dominic Orsi, Ben Puryear\\Section 1}
\date{November 17th, 2023}

\begin{document}

\maketitle

\section{Show that $A$ is context free. You will need a definition and a construction to do this. \\ $A = \{a^m b^n c^n | m, n \ge 0\}.$}

Definition of context free language: \\
A language is context free if and only if there is a PDA that recognizes it. \\
CFL's are generated by context free grammars.
\\
\\
Construction of context free grammar: \\
$
    S \Rightarrow XY \\
    X \Rightarrow aX|\epsilon \\
    Y \Rightarrow bYc|\epsilon
$
\\
\\
Construction of a PDA that accepts $A$: \\
$Q = \{q_0, q_1, q_2, q_3, q_4, q_5\} \\$
$\Sigma = \{a, b, c\} \\$
$\Gamma = \{b, \$\} \\$
$q_{\text{start}} = q_0 \\$
$F = \{q_5\} \\$
$\delta =$
\begin{tabular}{ |c|c|c| }
    Input      & Stack      & Rule                                \\
    \hline
    $\epsilon$ & $\$$       & $\epsilon, \epsilon \Rightarrow \$$ \\
    $b$        & $b\$$      & $b, \epsilon \Rightarrow 0$         \\
    $a$        & $\$$       & $a, \epsilon \Rightarrow \epsilon$  \\
    $c, b$     & $b\$$      & $c, b \Rightarrow \epsilon$         \\
    $\$$       & $\epsilon$ & $\epsilon, \$ \Rightarrow \epsilon$ \\
\end{tabular}

\pagebreak

\section{Show that $B$ is context free. You will need a definition and a construction to do this. \\ $B = \{a^n b^n c^m | m, n \ge 0\}.$}

Definition of context free language: \\
A language is context free if and only if there is a PDA that recognizes it. \\
CFL's are generated by context free grammars.
\\
\\
Construction of context free grammar: \\
$
    S \Rightarrow XY \\
    X \Rightarrow aX|\epsilon \\
    Y \Rightarrow bYc|\epsilon
$
\\
\\
Construction of a PDA that accepts $A$: \\
$Q = \{q_0, q_1, q_2, q_3, q_4, q_5\} \\$
$\Sigma = \{a, b, c\} \\$
$\Gamma = \{a, \$\} \\$
$q_{\text{start}} = q_0 \\$
$F = \{q_5\} \\$
$\delta =$
\begin{tabular}{ |c|c|c| }
    Input         & Stack      & Rule                                \\
    \hline
    $\epsilon$    & $\$$       & $\epsilon, \epsilon \Rightarrow \$$ \\
    $c, \epsilon$ & $\$$       & $c, \epsilon \Rightarrow \epsilon$  \\
    $a$           & $\$$       & $a, \epsilon \Rightarrow \epsilon$  \\
    $b, a$        & $\$a$      & $b, a \Rightarrow \epsilon$         \\
    $\$$          & $\epsilon$ & $\epsilon, \$ \Rightarrow \epsilon$ \\
\end{tabular}

\pagebreak

\section{}

\subsection{Let $C = A \cap B$. Using set notation, describe $C$.}

$C = \{a^n b^n c^n | n \ge 0 \}$.

\subsection{Use the results from problems 1, 2, and examples 2.36 in Sipser, to prove that the class of context-free languages is not closed under intersection.}

Assume $C$ is context free. The pumping length, $p$, is $n$. Therefore $C = a^p b^p c^p$. \\
However, this contradicts the first property of CFL (for each $i \ge 0, u v^2 x y^2 z \in A$) because it is impossible to describe an equal amount of $a, b,$ and $c$.

\end{document}