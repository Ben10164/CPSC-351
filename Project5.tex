\documentclass{article}
\usepackage{amsmath}
\usepackage{amssymb}
\usepackage{pgf}
\usepackage{tikz}
\usetikzlibrary{arrows,automata}
\usepackage[latin1]{inputenc}
\usepackage{lipsum}       % for sample text
\usepackage{changepage}   % for the adjustwidth environment

\newcommand{\lambdabar}{{\mkern0.75mu\mathchar '26\mkern -9.75mu\lambda}}

\title{Project 5}
\author{Jesse Adams, Dominic Orsi, Ben Puryear}
\date{October 12th, 2023}

\begin{document}

\maketitle

\section{Prove the following lemma by PMI. $\text{If } u,v \in \Sigma^* \text{, then }(uv)^R =v^Ru^R$.}

Basis:

If $|v|$ and $|u|$ are $0$, $v \in \Sigma^*$,

then $u = \lambda$

s.t.
\begin{align*}
    (uv)^R & = (\lambda v)^R   \\
           & = (v)^R           \\
           & = v^R (\lambda)^R \\
           & = v^R \lambda^R   \\
           & = v^R u^R
\end{align*}
\newline
Inductive Step:

Let $|u| = n > 0$

assume $u = aw$, where $w$ is a string, $|w| = n - 1$ and $a \in \Sigma$

Then,

\begin{align*}
    (uv)^R & = (awv)^R                        \\
           & = (awv)^R                        \\
           & = v^R (aw)^R                     \\
           & =v^R w^R (a)^R                   \\
           & =v^R w^R a^R \lambda^R           \\
           & =v^R w^R a^R \lambda v^R w^R a^R \\
           & =v^R (wa)^R                      \\
           & =v^R u^R
\end{align*}
We have now proven that $(uv)^R = v^R u^R$ through mathematical induction.

\section{Using induction on the length of the string, and the lemma you proved in problem 1, prove that $(w^R)^R = w \text{ for all strings } w \in \Sigma ^*$.}

Let $w = uv$
\begin{align*}
    ((w)^R)^R & = ((uv)^R)^R                      \\
              & = (v^R u^R)^R                     \\
              & = u^R (v^R)^R                     \\
              & = u^R v^R                         \\
              & = (uv)^R \ [\text{From Q1 Lemma}] \\
              & =vu                               \\
              & = w
\end{align*}

We have now proven that $((w)^R)^R = w$ through mathematical induction.
\section{Let $J$ be the set of palindromes over $\{a,b\}$.}

\subsection{Give a formal definition of $J$ using set notation.}

$J = \{w | w \in \{a \cup b\}^* \text{ s.t. } w^R = w\}\}$

\subsection{Using the pumping lemma for regular languages, show that $J$ is not regular.}

The pumping lemma can be used to determine if $J$ is a regular language by determining if there is a number $P$, the pumping length, where if $s$ is any string in $L$ of length at least $P$, then $s$ may be divided into 3 parts, $S=xyz$ such that: \\
1. for each $i \ge 0$, $xy^iz \in J$ \\
2. $|y| = 0$ \\
3. $|xy| \le P$


In this case, due to $J$ being the set of palindromic strings formed from $\{a, b\}$, we can define $S = a^p b^p a^p$.


Through the first criteria for the modified version of $s$ used in the pumping lemma, $\forall i > 0, xy^iz \in J$, we can deduce the following.

If $S = a^h a^i a^{p-h-i} b^p a^p \in J$, then $S' = a^h a^i a^i a^{p-h-i} b^p a^p \in J$ which is equivalent to $a^{i+p} b^p a^p \in J$.

However, when defining this, we specified that $i > 0$, which is contradicted by this statement. \\
Therefore: \\
Since $i \not = 0$, $S'^R \not = S'$, so $S' \not \in J$. \\
This means that $S$ is not a regular language.


\section{Let $L = \{ww | w \in \{a,b\}* \}$. Using the pumping lemma for regular languages, show that L is not regular.}

\textbf{Lets assume that L is regular: }
\\
Consider the constant $k$ ($k \ge 1$)
\\
Consider a string $s = a^mb^mb^ma^m$ where $m \ge 1$
\\
Now consider decomposition's for $s$.
\\
They will be of the form $s=xy$, where $y=a^j$ where $1 \le j \le k$.
\\
Now by pumping lemma, $k v^i k$ belongs to $L$ for all $i \ge 0$.
\\
Consider $i=0$, then we have a string in the form $a^l b^m a^m b^m$ where $l < m$.
\\
Since the number of $a$'s on the left side of the string is less than $a$'s on the right side of the string, they can never be represented as $ww$, where $w \in ab^*$.
\\
This \textbf{violates} our assumption that $L$ is regular.
\\
So by \textbf{contradiction}, we have shown that $L$ is not regular.
\end{document}